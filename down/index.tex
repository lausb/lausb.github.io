\documentclass[12pt]{amsproc}
\usepackage[colorlinks,urlcolor=blue,citecolor=blue,linkcolor=blue]{hyperref}
\usepackage{graphicx,anysize,multicol}
\marginsize{2cm}{2cm}{2cm}{2cm}
\renewcommand{\contentsname}{Shibo Liu's Homepage}
\setcounter{tocdepth}{2} 

\renewcommand{\refname}{}

\begin{document}


\begin{multicols}{3}
\tableofcontents

\end{multicols}

\hrule

Dr. Shibo Liu is an assistant professor of mathematics at Florida Tech.

\begin{itemize}
\item Email:  \href{mailto:sliu@fit.edu}{\texttt{sliu@fit.edu}}
\item \href{https://www.fit.edu/faculty-profiles/l/liu-shibo/}{Official Website}
\item \href{https://scholar.google.com/citations?hl=en&user=yEjKTkIAAAAJ}{Google Scholar}
\item \href{https://mathscinet.ams.org/mathscinet/search/author.html?mrauthid=671940}{MathSciNet} (Amer. Math. Soc.)
\item \href{https://app.box.com/s/8w8596ssl2yvc5207fw9cjok6huv16gh}{My Downloads}
\end{itemize}

Update: \today

\dotfill

\section{Research Interests}
\begin{itemize}
\item Nonlinear Functional Analysis
\item Variational Methods, Critical Point Theory
\item Nonlinear Elliptic Partial Differential Equations
\item Differential Geometry
\item Mathematical Analysis
\end{itemize}

Welcome to \href{mailto:sliu@fit.edu}{contact me} if you are interested to work with me for your MSc or PhD degree. 

\section{Professional Experiences}

\begin{itemize}
\item 08.2022--Present:  \href{https://www.fit.edu/mathematical-sciences/}{Department of Mathematical Sciences}, \href{http://www.fit.edu}{Florida Institute of Technology}, \emph{Assistant Professor}
\item 10.2011--07.2022: \href{http://math.xmu.edu.cn}{Department of Mathematics}, \href{http://www.xmu.edu.cn}{Xiamen
University}, \emph{Professor}
\item 08.2008--08.2011: \href{http://math.stu.edu.cn}{Department of Mathematics}, \href{http://www.stu.edu.cn}{Shantou University}, \emph{Professor}
\item 07.2005--07.2008: \href{http://math.xmu.edu.cn}{Department of Mathematics}, \href{http://www.xmu.edu.cn}{Xiamen
University}, \emph{Associate Professor}
\item 07.2003--06.2005: \href{https://www.math.pku.edu.cn/xygk/xssys/sxyjs1/69260.htm}{Institute of Mathematics}, \href{http://www.pku.edu.cn}{Peking
University}, \emph{Postdoc}
\end{itemize}\medskip

\begin{itemize}
\item 03.2020--08.2020: \href{http://www.xmu.edu.my}{Xiamen University
Malaysia}, \emph{Visiting Professor}
\item 01.2017--12.2017: \href{http://www.nd.edu}{University of Notre Dame}, \emph{Guest Professor}
\item 01.2014--12.2019: \href{http://www.ictp.it}{Abdus Salam International Centre for Theoretical Physics (ICTP)},\\
\emph{Regular Associate member}
\item 02.2013--06.2018: \href{http://math.xmu.edu.cn}{Department of Mathematics}, \href{http://www.xmu.edu.cn}{Xiamen
University}, \emph{Associate Dean}

\end{itemize}

\section{Educational Background}

\begin{itemize}
\item 09.2000--06.2003: \href{http://www.amss.ac.cn}{Institute of Mathematics}, \href{http://www.cas.ac.cn}{Chinese Academy of
Sciences}, Ph.D.\\ Supervisor: \emph{Prof. \href{https://mathscinet.ams.org/mathscinet/search/author.html?mrauthid=226307}{Shujie Li}}
\item 09.1997--06.2000: \href{http://math.lzu.edu.cn}{Department of Mathematics}, \href{http://www.lzu.edu.cn}{Lanzhou
University}, M.Sc.\\ Supervisor: \emph{Prof. \href{https://mathscinet.ams.org/mathscinet/search/author.html?mrauthid=205096}{Xianling Fan}}
\item 09.1993--06.1997: \href{http://math.lzu.edu.cn}{Department of Mathematics}, \href{http://www.lzu.edu.cn}{Lanzhou
University}, B.Sc.
\end{itemize}

\section{Teaching}

\begin{itemize}
\item Spring 2023: Intro to PDE \& Apps (MTH 3210)
\item Fall 2022: Intro to PDE \& Apps (MTH 3210)\\
\phantom{Fall 2022: }Calculus 3 (MTH 2001)
\item Spring 2020: Mathematical Analysis II, \href{http://www.xmu.edu.my}{Xiamen University
Malaysia}
\item Spring 2020: Ordinary Differential Equations, \href{http://www.xmu.edu.my}{Xiamen University
Malaysia}
\item Before I came to USA in July 2022, I taught various courses in China for undergraduate and graduate students majoring in mathematics, such as:
\begin{multicols}{2}
\begin{itemize}
\item Mathematical Analysis\\
\item Real Analysis\\
\item Partial Differential Equations\\
\item Differential Geometry\\
\item Nonlinear Functional Analysis
\end{itemize}
\end{multicols}


\end{itemize}

\section{Research}

\subsection{Selected Publications}

Click \href{https://mathscinet.ams.org/mathscinet/search/author.html?mrauthid=671940}{MathSciNet} (American Mathematical Society) for the complete list of my publications.

\bibliographystyle{year}
\nocite{*}
\bibliography{lau}

\subsection{Research Grants}

As Principle Investigator, I have conducted 4 research projects funded by \href{https://www.nsfc.gov.cn}{Natural Science Foundation of China}.
\begin{itemize}
\item Quasilinear Schr\"{o}dinger equations with indefinite potential and
related problems, \emph{National Natural Science Foundation of China} grant
(12071387), 01.2021--12.2024. Principal Investigator

\item Standing waves for nonlinear Schr\"{o}dinger-Poisson systems with high
frequency and related problems, \emph{National Natural Science Foundation of China} grant
(11671331), 01.2017--12.2020. Principal Investigator
\item Critical point theory and nonlinear Schr\"{o}dinger-Poisson systems, \emph{Distinguished Young Scholar Foundation of Fujian} grant (2014J06002), 01.2014--12.2016. Principal Investigator

\item Strongly indefinite and noncompact variational problems, \emph{National Natural Science Foundation of China} grant
(11171204), 01.2012--12.2015. Principal Investigator

\item Critical groups and multiple solutions of nonlinear differential
equations, \emph{National Natural Science Foundation of China} grant
(10601041), 01.2007--12.2009. Principal Investigator
\item Program for New Century Excellent Talents in Fujian Province
University, 1.2008--12.2010. Principal Investigator
\end{itemize}

\section{Selected Presentations}

\begin{itemize}
\item Morse theory and nonlinear Schrodinger equations, \emph{Florida Institute of Technology}, Jan 20, 2023
\item \href{http://tianyuan.xmu.edu.cn/cn/letures/1010.html}{Surjections between Euclidean Spaces, Changing Variables and Brouwer Fixed Point Theorem}, \emph{Tianyuan Mathematical Center in Southeast China} (TMSE), Dec 11, 2022\\
(\href{https://lausb.github.io/math/tmse.pdf}{Slides})
 
\end{itemize}

\section{\LaTeX{} \& Web-Pages}

Here are some templates for \LaTeX{} typesetting:
\begin{itemize}
\item \href{https://lausb.github.io/down/NSFC.zip}{NSFC Application}
\item \href{https://lausb.github.io/down/meet.zip}{Booklet for Conference}
\item Slides for Academic Presentation
\item Universal Thesis (In Chinese, suitable for \textbf{ANY} Institution)
\end{itemize}

\LaTeX{} can also be used to build your website:
\begin{itemize}
\item Write your webpage via \LaTeX{} in a file named \verb|index.tex|. \href{https://lausb.github.io/down/index.tex}{Here} is the \TeX{} file I used to create this website.
\item In the directory of \verb|index.tex|, run


 \verb|make4ht index.tex "mathjax"|
  
in \emph{Command Line} of your system. This will generate \verb|index.html| and \verb|index.css|, which will be uploaded to your site
 
See \href{https://tug.org/tex4ht}{here} for more commands you can use.  
 You need to have full installation of \TeX{} system, such as TexLive
\item Follow the steps \href{https://docs.github.com/en/pages/quickstart}{here} to setup your \verb|GitHub| website and upload \verb|index.html| and \verb|index.css| to the site
\item Now, enjoy your website. You may display mathematics on your webpages, for example
\[
\chi(M)=\frac{1}{2\pi}\int_MK\,\mathrm{d}\sigma\text{.}
\]

\end{itemize}


\end{document}
